\documentclass{beamer}
\usepackage[utf8]{inputenc}
\usepackage{graphicx}
\usetheme{Madrid}
\usepackage{amsmath}
\usepackage{amssymb}

\graphicspath{../SciCompFinalProject}

\newcommand{\pth}[1]{\left(#1\right)}
\newcommand{\brc}[1]{\left\{#1\right\}}
\newcommand{\bra}[1]{\left[#1\right]}
\newcommand{\dd}{\text{d}}
\newcommand{\eps}{\varepsilon}

\title[Scientific Computing WiSe 2021/22]{The Bidomain problem}
\subtitle{A Julia implementation with VoronoiFVM.jl}
\author[{\fontsize{5.5}{6}\selectfont Henry Jacobson \& Santos Michelena \& Ryan Quinn  \hspace{5mm}}]{Henry Jacobson \\ Santos Michelena \\ Ryan Quinn} %see if theres a better way to do this, this is awful
\institute{Technische Universität Berlin}
\date[Prof. Dr. Jürgen Fuhrmann]{March 2022}

\begin{document}

\frame{\titlepage}

\begin{frame}
	\frametitle{Contents}
	\tableofcontents
\end{frame}

\section{Problem introduction}

\begin{frame}
	\frametitle{What is the bidomain problem?}
	
	The bidomain problem is a system of partial differential equations modeling the propagation of electric signals throughout cardiac tissue. \cite{intro} According to \cite{paper}, this system is given by:
	
	\begin{block}{Bidomain problem}
		\begin{align*}
			\frac{\partial u}{\partial t} &= \frac{1}{\eps}f(u,v)+
			\nabla\cdot(\sigma_i\nabla u)+\nabla\cdot(\sigma_i\nabla u_e)\\
			0&=\nabla\cdot(\sigma_i\nabla u+(\sigma_i +\sigma_e)\nabla u_e)\\
			\frac{\partial v}{\partial t} &= \eps g(u,v).
		\end{align*}
	\end{block}
	
	
	
\end{frame}



\begin{frame}
	\frametitle{Bibliography}
	
	\begin{thebibliography}{3}
		\bibitem{paper} 
		Ethier, Marc; Bourgault, Yves Semi-Implicit Time-Discretization Schemes For The Bidomain Model, Siam Journal On Numerical Analysis (2008)
		
		\bibitem{intro}
		Hooke, N.; Henriquez, C.S.; Lanzkron, P.; Rose, D. Linear Algebraic Transformations Of The Bidomain Equations: Implications For Numerical Methods, Mathematical Biosciences (1994)
		
		\bibitem{library}
		Fuhrmann, Jürgen Voronoifvm.Jl: Finite Volume Solver For Coupled Nonlinear Partial Differential Equations, Zenodo (2022)
	\end{thebibliography}
	
\end{frame}



\end{document}
